\documentclass[pdf,hyperref={unicode}, aspectratio=43, serif,11pt]{beamer}
\usepackage[T2A]{fontenc}
\usepackage[english, russian]{babel}  
\usepackage{listings}
\usepackage{subcaption}

\usepackage{color,colortbl}
\definecolor{Gray}{RGB}{211,211,211}
           
%Красная строка в первом абзаце
\usepackage{indentfirst}
%Величина отступа красной строки
\setlength{\parindent}{12.5 mm}

\usepackage{graphicx}

\usepackage{setspace}
\setstretch{1}

\title[Физико-математический факультет]{Орловский государственный
университет имени И.\,С.~Тургенева}
\author{Гайворонцев М.Р.}
\institute[]{Орловский государственный
университет имени И.\,С.~Тургенева}
\def\baselinestretch{1}

\usefonttheme[onlymath]{serif}
\usepackage{beamerthemesplit}

%тема оформления
\usetheme{Madrid}%Warsaw
%цветовая гамма
\usecolortheme{whale}%whale


\begin{document}

\begin{frame}
\titlepage
\end{frame}


\begin{frame}{Введение: Физмат – Кузница Знаний}
    \begin{columns}
        \column{0.5\textwidth}
        \includegraphics[width=\textwidth]{}

\begin{figure}
            \centering
            \includegraphics[width=1\linewidth]{}
\begin{figure}
                \centering
                \includegraphics[width=0.7\linewidth]{666.png}
                \label{fig:sub1}
            \end{figure}
            \label{fig:enter-label}
        \end{figure}

        \column{0.5\textwidth}
        \begin{itemize}
            \item Физико-математический факультет — один из старейших в ОГУ. Он сыграл ключевую роль в развитии образования. Факультет стал центром подготовки высококвалифицированных специалистов.
            \item Академическая строгость: Факультет известен глубиной изучения фундаментальных наук. Здесь обучают логическому мышлению.
        \end{itemize}
    \end{columns}
\end{frame}

\begin{frame}
\frametitle{\textbf{Основные Исторические Этапы}}
\begin{itemize}
    \item {\small 1931 год - Основание Орловского индустриально-педагогического института. Заложено начало Физмата.}
    \item {\small 1930-е годы - Формирование кафедр физики и математики. Разработка первых учебных планов.}
    \item {\small Великая Отечественная война - Работа института приостановлена. Многие преподаватели и студенты ушли на фронт.}
    \item {\small 1950-1960-е годы - Восстановление и расширение факультета. Открытие новых специальностей.}
    \item {\small 1990-е годы - Переход к рыночной экономике. Адаптация программ обучения.}
    \item {\small 2000-е годы - Модернизация образовательного процесса. Внедрение инновационных технологий.}
\end{itemize}
\end{frame}

\begin{frame}
\frametitle{Известные Личности}
\begin{center}
    \includegraphics[scale=0.38]{777.jpg}
\end{center}
\end{frame}

\begin{frame}
\frametitle{\textbf{Хронология Ключевых Событий}}
\begin{itemize}
{\small Важные даты в истории Физико-математического факультета и его развития.}
    \item {\small 1931 год - Основание института, включившего физмат-отделение}
    \item {\small 1941-1945 - Приостановка работы из-за войны}
    \item {\small 1953 - Открытие аспирантуры по физико-математическим наукам}
    \item {\small 1970-е - Развитие компьютерных технологий}
    \item {\small 1994 - Переименование в Орловский государственный университет}
    \item {\small 2016 - Получение статуса опорного университета}
\end{itemize}
\end{frame}



\begin{frame}
\frametitle{Интересные Факты и Достижения}
\begin{table}[H]
    \centering
    \begin{tabular}{|p{4cm}|p{7.1cm}|}
        \hline
        \cellcolor {Gray} \textbf{Факты и Достижения} & \cellcolor {Gray} \textbf{Описание} \\
        \hline
        Высокий рейтинг & Факультет стабильно входит в число лучших физмат-факультетов России. \begin{center}
            \includegraphics[scale=0.18]{}
        \end{center}\\
         \hline
        Олимпийские победы & Студенты Физмата регулярно побеждают на всероссийских олимпиадах по физике и математике.
        \begin{center}
            \includegraphics[scale=0.18]{}
        \end{center}\\
        \hline
        Социальные проекты & Факультет активно участвует в просветительской деятельности. Поддержка молодых талантов.
        \begin{center}
            \includegraphics[scale=0.18]{}
        \end{center}\\
    \end{tabular}
    \label{tab:my_label}
\end{table}
\end{frame}


\begin{frame}{Современный физмат}
    \begin{columns}
        \column{0.4\textwidth}
        \includegraphics[width=\textwidth]{2.jpg}
        
        \column{0.6\textwidth}
        \begin{itemize}
            \item 7 кафедр
            \item 634 студента (2007)
            \item 8 докторов наук
            \item Компьютерные классы
            \item Спортивные традиции
        \end{itemize}
    \end{columns}
    
    \vspace{0.5cm}
    \centering
\end{frame}

\begin{frame}
\frametitle{Заключение и Перспективы}
{\small История Физмата — это путь к знаниям. Факультет продолжает развиваться, открывая новые горизонты.}
\begin{itemize}
    \item {\small \href{https://phys-math.ru/history/start}{Узнать больше}}
    \item {\small \href{https://oreluniver.ru}{Сайт ОГУ}}
\end{itemize}
\end{frame}


\end{document}